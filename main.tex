\documentclass[dvipdfmx,11pt,a4paper]{jsbook}
\usepackage{package}
%\usepackage{a4wide}
\usepackage[dvipdfmx]{hyperref}
\usepackage{pxjahyper}


%\title{}
%\author{}
%\date{\today}
\begin{document}
%\maketitle

%\tableofcontents

%\makeatletter
%\@addtoreset{equation}{section}
%\def\theequation{\thesection.\arabic{equation}}
%\makeatother

\addtolength{\fullwidth}{-26truemm} %全体の幅(ヘッダ部の幅)を既定値から26mm小さくする
\setlength{\textwidth}{\fullwidth}  %本文の幅(textwidth)を全体の幅(=ヘッダ部の幅)にそろえる
\setlength{\evensidemargin}{10truemm}   %偶数ページの左余白を10mm(+1インチ)にする
\setlength{\oddsidemargin}{10truemm}    %奇数ページの左余白を10mm(+1インチ)にする
\chapter{Preface cum introduction}

\chapter{CLASSICAL SOLITONS AND SOLITARY WAVES}
\begin{itembox}[l]{目的}
    非線形方程式の古典解のうち,ミンコフスキー計量やユークリッド方程式における場の方程式に対応するものがいくつかあり,そういった古典解から相対論的量子場の理論の情報を得る.
\end{itembox}

\section{Introduction}

本ゼミにおいてのメインテーマとなるソリトンとインスタントンであるが,どちらも簡単にいえば形状を保ったまま進行し互いに衝突・追突しても崩れないような局在化(localized)した波の事である.そもそもソリトンの英語スペルはsolitonであり,これはsolitary(孤立した)+on(粒子につける接尾辞(Fermi{\bf on}, Bos{\bf on}, Glu{\bf on}, Phot{\bf on} \dots))から来ている.すなわち孤立した波の塊でありながら歪むことなく安定に一様な速度で進行する波であり,粒子的に振る舞うような物理的対象を指す.ここで,なぜ局在化した波が粒子としてみなせるかということについては場の理論において場が作る波をエネルギーのたまり場のようなものであると捉えると,ポツンと局在化したエネルギーが非離散的(連続的)に形を崩さずに移動していればまるで粒子が移動していると拡大解釈できることから理解できる.

このようなソリトンは様々な応用が効く.本来素粒子は量子論で語られるべきものであるが,相対論的な場の理論から素粒子もlocalizeされたエネルギーパケットをもっていることがわかっており,古典場の理論からスタートするソリトンを用いてその振る舞いを説明することができ,ソリトンが古典論と量子論をつなげる可能性を持ち合わせていることを意味する.他にもソリトンには原子核,陽子中性子の模型や宇宙論のテクスチャ,物性の磁気スカーミオンなど素粒子原子核から物性,宇宙まで様々な分野において重要な役割を果たす.

\begin{figure}[H]
    \centering
    \includegraphics[width=8cm]{figure/soliton.png}
    \caption{ソリトン}
    \label{soliton}
\end{figure}
本章におけるメインテーマはsin-Gordons solitonやkinkの$\phi^4$理論や'tHooft-Polyakov monopoleやインスタントンなどの例を用いながらソリトン(Solitons)と孤立波(Solitary waves)の区別をつけるということである.

また,以下にこれから使う記法についてまとめておく.
\begin{screen}
    \begin{itemize}
        \item 空間と時空の座標系はベクトル$x^{\mu}(\mu=0, 1, 2, 3, ; x^{0}\equiv ct, x^{1}=x, x^{2}=y, x^{3}=z)$で与えられる.
        \item 上付きまたは下付きの添字はミンコフスキー計量テンソル$g_{\mu \nu}=g^{\mu \nu}$;
              \begin{align*}
                  g_{\mu \nu}=
                  \left(\begin{array}{cccc}
                          1 & 0  & 0  & 0  \\
                          0 & -1 & 0  & 0  \\
                          0 & 0  & -1 & 0  \\
                          0 & 0  & 0  & -1
                      \end{array}\right)
              \end{align*}
              で書かれる.(Mostly minus)
        \item 同じ添字がついてるものについては足し上げる.(縮約記法)
        \item $\partial_{\mu}$は時間や空間での微分$\frac{\partial}{\partial_{\mu}}$を表す.
        \item 簡単のために本セクションに限っては1つの時間座標と1つの空間座標の(1+1)次元,つまり$\mu=0,1$とする.
    \end{itemize}
\end{screen}

\section{Solitary waves and solitons}
先に述べたようにSolitonとSolitary waveは非線形の波動方程式においてどちらも局在化した特別な特徴をもつ解でありその区別はどのようにできるかを学ぶ.

\subsection{Case of the simplest wave equation}
まず,よくある最もシンプルな線形かつ非分散的である波動方程式
\begin{align}
    \Box\phi & =\partial^{\mu}\partial_{\mu}\phi\nonumber                                                                             \\
             & =\left(\dfrac{1}{c^2}\dfrac{\partial^2}{\partial t^2}-\dfrac{\partial^2}{\partial x^2}\right)\phi(x,t)=0\label{eq:2.1}
\end{align}
を考える.この解は以下のような2つの特徴をもつ.
\begin{description}
    \item[(i)] 解である$f(x\pm ct)$は全領域で一意に定まり,微分が定義され,連続である(well-behaved function).また,$\omega=kc$とすれば$\cos(kx\pm \omega t)$と$\sin(kx\pm \omega t)$の平面波はどちらも式(\ref{eq:2.1})の1つの解となっておりフーリエ変換で重ね合わせれば
          \begin{align}
              f(x-c t)=\int \mathrm{d} k\left[a_{1}(k) \cos (k x-\omega t)+a_{2}(k) \sin (k x-\omega t)\right]
          \end{align}
          とかける.そして仮に局在化した解$f(x-ct)$をもってきた場合,その波のパケットは一定速度$\pm c$で歪みや形の崩れなく進む.
    \item[(ii)] 波動方程式が線形であるが故に解として波のパケット$f_1(x-ct)$と$f_2(x+ct)$が与えられれば,それらの線形結合である
          \begin{align*}
              f_3(x,t)=f_1(x-ct)+f_2(x+ct)
          \end{align*}
          もまた解となる.

          そして$t\rightarrow-\infty$で$f(x,t)$は2つのパケットはそれぞれ別の方向に向かって二分され,その後時間をかけて互いに歪まずに接近し衝突する.そして衝突後も形を崩さず$t\rightarrow\infty$に向かって同じ2つのパケットが一定の速度で移動していく.
\end{description}
つまりこれらを簡単にまとめれば式(\ref{eq:2.1})の解は以下の性質を持つ.
\begin{screen}
    \begin{description}
        \item[(i)] 1つのパケットが歪まず,形を維持したまま一定速度で移動する.
        \item[(ii)] いくつかのパケットが互いに衝突したとしてもその進行速度と形が維持される.(別ファイルに添付したGifを参照.)
    \end{description}
\end{screen}
\begin{figure}[H]
    \centering
    \subfigure[]{%
        \includegraphics[clip, width=0.45\columnwidth]{figure/2-soliton1.png}}%
    \subfigure[]{%
        \includegraphics[clip, width=0.45\columnwidth]{figure/2-soliton2.png}}%
    \label{2-soliton}
\end{figure}

\subsection{Case of Klein-Gordon equation}
しかしながら大半の波動方程式は非線形校を含んでいたり,分散的であったりと複雑で全ての波動方程式が性質(i),(ii)を満たしているとは限らない.

そのうちの一つが式(\ref{eq:2.1})に$m^2c^2$の項を加えたKlein-Gordon方程式
\begin{align}
    \left(\square+m^{2} c^{2}\right) \phi(x, t) & \equiv\left(\frac{1}{c^{2}} \frac{\partial^{2}}{\partial t^{2}}-\frac{\partial^{2}}{\partial x^{2}}+m^{2} c^{2}\right) \phi(x, t)\nonumber \\
                                                & =0
\end{align}
がある.これは明らかに先と同様に線形でありながら$\cos(kx\pm \omega t)$と$\sin(kx\pm \omega t)$の平面波解をもつ.しかし今回の場合は$\omega^2=k^2c^2+m^2c^4$であり分散的という違いがある.いま先と同様に線形であることを用いて局在化したパケット解
\begin{align*}
    f(x-ct)=\int \mathrm{d} k\left[a_{1}(k) \cos (k x-\omega t)+a_{2}(k) \sin (k x-\omega t)\right]
\end{align*}
を作ったときに今回の場合,波の進行速度が$\omega(k)/k$というように波数$k$に依存することになる.したがって一度作った波のパケットが時間とともに違った速度で形を広げながら伝搬してしまい,先に述べた(i)(ii)の特徴を持ち合わせていないということがわかる.(別ファイルに添付したGifを参照.)
\begin{figure}[H]
    \centering
    \subfigure[]{%
        \includegraphics[clip, width=0.53\columnwidth]{figure/wave(spread)1.png}}%
    \subfigure[]{%
        \includegraphics[clip, width=0.55\columnwidth]{figure/wave(spread)2.png}}%
    \label{wave(spread)}
\end{figure}
なお紹介だけにはなるが,他には解析的には解けないが式(\ref{eq:2.1})に$\phi^3$の項を付け加えたような以下の非線形波動方程式も(i)(ii)の特徴を持ち合わせていない.
\begin{align}
    \left(\frac{1}{c^{2}} \frac{\partial^{2}}{\partial t^{2}}-\frac{\partial^{2}}{\partial x^{2}}\right) \phi(x, t)+\phi^{3}(x, t)=0\label{eq:2.4}
\end{align}

\subsection{View of energy density}
先の議論を踏まえて一般的にはSolitonとSolitary-waveの以下のような区別がなされている.
\begin{screen}
    \begin{itemize}
        \item (i)の特徴を満たすもの $\cdots$ {\bf Solitary-wave}, {\bf Soliton}
        \item (i)を及び(ii)の特徴を満たすもの $\cdots$ {\bf Soliton}
    \end{itemize}
\end{screen}

しかしこれではどういったときに(i)または(ii)の特徴を満たすのかはわかっておらず,普遍的ではない.

ここで場$\phi_i$の波自体に着目するのではなく,むしろ場$\phi_i$がつくるエネルギー密度$\epsilon$に着目してみる.まず条件としては以下のような設定を課す.
\begin{itemize}
    \item 空間内の全エネルギー$E[\phi_i]$は保存し,一般性を失わない.
    \item エネルギー$E$の最小値は0.
    \item 空間内でエネルギーは有限の値をとるが無限遠では$E=0$.
    \item 任意の時刻$t$において十分遠方($x\rightarrow \infty$)で最小値$E[\phi_i]$に漸近するような場の方程式の解に対して局在化した(localized)という言葉をつかう.
\end{itemize}
ではこれらをいくつかの例を紹介しながらlocalizeするとはどういうことか見ていく.先に紹介した式(\ref{eq:2.4})の方程式
\begin{align*}
    \left(\frac{1}{c^{2}} \frac{\partial^{2}}{\partial t^{2}}-\frac{\partial^{2}}{\partial x^{2}}\right) \phi(x, t)+\phi^{3}(x, t)=0
\end{align*}
においては場$\phi$が作るエネルギーを見てみると
\begin{align}
    E[\phi]=\int_{-\infty}^{\infty} \mathrm{d} x\left[\frac{1}{2 c^{2}}\left(\frac{\partial \phi}{\partial t}\right)^{2}+\frac{1}{2}\left(\frac{\partial \phi}{\partial x}\right)^{2}+\frac{1}{4} \phi^{4}\right]
\end{align}
で与えられる.これは$\phi=0$で最小値$E[\phi]$をとるから,localizeすると言いたければ,十分遠方$x\rightarrow \infty$で$\phi(x,t)\rightarrow 0$となっていれば良い.\\

他には自発的対称性の破れを記述する有名な
\begin{align}
    \frac{1}{c^{2}} \frac{\partial^{2} \phi}{\partial t^{2}}-\frac{\partial^{2} \phi}{\partial x^{2}}-\phi+\phi^{3}=0
\end{align}
という非線波動方程式におけるエネルギーは
\begin{align}
    E[\phi]=\int_{-x}^{\infty} \mathrm{d} x\left[\frac{1}{2 c^{2}}\left(\frac{\partial \phi}{\partial t}\right)^{2}+\frac{1}{2}\left(\frac{\partial \phi}{\partial x}\right)^{2}+\frac{1}{4}\left(\phi^{2}-1\right)^{2}\right]
\end{align}
で与えられる.この$E[\phi]$は$\phi(x,t)=\pm 1$で最小値をとるから,localizeすると言いたければ,十分遠方$x\rightarrow \infty$で$\phi(x,t)\rightarrow \pm 1$となる必要がある.

\subsubsection{Solitary wave}
localizeするとはどういうことかが普遍的に述べられた上でエネルギー密度の観点からSolitary waveとは何かを考える.Solitary waveは場の方程式においてlocalizeしているという特異性だけでなく次のような時空依存のエネルギー密度を描くものと定義できる.
\begin{align}
    \epsilon(\bm{x},t)=\epsilon(\bm{x}-\bm{u}t)\quad (\bm{u}: \text{速度ベクトル})\label{eq:2.8}
\end{align}
これは明らかに場が作るエネルギー密度のパケットが一定の速度で形が変わらずに移動するという(i)の特徴を表現できており,これがまさにSolitary  waveの定義とできる.

\subsubsection{Soliton}
次にSolitonについてもエネルギー密度の観点からその定義を考える.Solitonは先の議論の特徴(i)だけでなく特徴(ii)に重きを置く必要がある.特徴(ii)はいくつかのパケットが互いに衝突したとしてもその進行速度と形が維持されるということであった.そこで以下のような状況設定で議論を進める.
\begin{itemize}
    \item 非線形波動方程式でlocalizeした$N$個のsolitary wave解$\epsilon_0(\bm{x}-\bm{u}t)$で解が構成されている.
\end{itemize}
上記の状況を式で表せば
\begin{align}
     & \varepsilon(x, t) \rightarrow \sum_{i=1}^{N} \varepsilon_{0}\left(x-a_{i}-u_{i} t\right), \quad \text { as } t \rightarrow-\infty\label{eq:2.9}             \\
     & \varepsilon(x, t) \rightarrow \sum_{i=1}^{N} \varepsilon_{0}\left(x-a_{i}-u_{i} t+\delta_{i}\right), \quad \text { as } t \rightarrow+\infty\label{eq:2.10} \\
     & \qquad \ \ \ \ \ \ \ \ \ \ \ \ \ \ \ \ \ \ \ \ \  \ \ \ \ \ \ \ \ \ \ \ \ \ \ \ \ (\bm{\delta_i}:\text{定ベクトル})\nonumber
\end{align}
となる.これは$t\rightarrow -\infty$であるいくつかのSolitary waveのエネルギーパケットをもった解が一定速度かつ歪みなく進んだのち$t\rightarrow \infty$で時間発展したとしても形が変わらないという特徴(ii)の描像を描けているといえる.つまりこのエネルギー密度の関係式を満たすものがSolitonであると定義できる.\\

以上でSolitonとSolitary waveの区別ができるようになった.そしてこの世界には定常解を含めてSolitary waveは割と存在する.しかしながらSolitonは式(\ref{eq:2.10})の要請の縛りが強くほとんど存在しないというのが現状である.そもそも非線形場の方程式に対してSolitary waveの要請式(\ref{eq:2.8})を確認するのも割と大変な作業となる.そして式(\ref{eq:2.9})(\ref{eq:2.10})の両方を満たしているかを確認するのはさらに労力が必要となる.(この本ではやりたきゃ一人で時間をかけてやってどうぞと記載してある.)また,物理でSolitonと使われるとき大体はSolitary waveも含めてそう呼ぶ慣習があり,重要なことはその違いを理解していることである.研究等なんらかの場合に厳密にSolitonであることを示す必要がある場合には式(\ref{eq:2.8})(\ref{eq:2.9})(\ref{eq:2.10})の要請を確かめられれば良い.一応逆散乱法やバックグラウンド変換などといったエレガントな方法も存在するということも紹介しておく.

\section{Some solitary waves in two dimensions}
\subsection{Static Solitary wave and boundary conditions}
解が時間依存していない定常解もSolitary waveであったからそのいくつか簡単な例について2次元系で見ていく.

以下のようなローレンツ不変なラグランジアン密度
\begin{align}
    \mathscr{L}(x, t)=\frac{1}{2}(\phi)^{2}-\frac{1}{2}\left(\phi^{\prime}\right)^{2}-U(\phi)
\end{align}
を考える.\footnote{プライム記号("$'$")は時間微分を表しドット記号("$\dot{ }$")は空間微分を表す.}なおここでは光速$c=1$とし,ポテンシャル$U(\phi)$は$\phi$の関数で最小値が0のものとする.作用の変分をとれば
\begin{align}
    \delta\left(\int dt \int_{-x}^{x}dx\mathscr{L}(x,t)\right)=0
\end{align}
となりこの内の一つから以下の波動方程式が導かれる.
\begin{align}
    \square \phi \equiv \ddot{\phi}-\phi^{''}=-\frac{\partial U}{\partial \phi}(x, t)\label{eq:2.13}
\end{align}
そしてエネルギーの関数は
\begin{align}
    E[\phi]=\int_{-\infty}^{\infty} \mathrm{d} x\left[\frac{1}{2}(\phi)^{2}+\frac{1}{2}\left(\phi^{\prime}\right)^{2}+U(\phi)\right]\label{eq:2.14}
\end{align}
となる.そして$U(\phi)$が$M(\geq1)$個の点で最小値0をとする.
\begin{align}
    U(\phi)=0 \quad \text { for } \quad \phi=g^{(i)} ; \quad i=1, \ldots, M\label{eq:2.15}
\end{align}
また,仮に場$\phi(x,t)$が定数であれば式(\ref{eq:2.14})のエネルギーも最小値$E[\phi]=0$を
\begin{align}
    \phi=g^{(i)} ; \quad i=1, \ldots, M
\end{align}
でとる.

いま我々は簡単のために定常解のSolitary waveを考えるから式(\ref{eq:2.14})において時間の項を消せば
\begin{align}
    \phi^{\prime \prime}(x) \equiv \frac{\partial^{2} \phi}{\partial x^{2}}=+\frac{\partial U}{\partial \phi}(x)\label{eq:2.17}
\end{align}
となる.

さらにSolitary waveは有限のエネルギーでlocalizeしたエネルギー密度をもっているという要請をこれに課す.前セクションで述べたが,localizeするには任意の時刻$t$において十分遠方($x\rightarrow \pm\infty$)で最小値$E[\phi_i]$に漸近する必要があった.(($\partial\phi/\partial x\rightarrow 0$も.))いま式(\ref{eq:2.15})で最小値をとることがわかっているから,localizeするという要請から以下の境界条件が成立する.
\begin{screen}
    \begin{align*}
        U(\phi)\rightarrow0 \quad \text{and} \quad \left(\frac{\partial\phi}{\partial x}\right)\rightarrow 0,\qquad \text{as}\quad x\rightarrow \pm \infty
    \end{align*}
\end{screen}

\subsection{Solve the equation (\ref{eq:2.17})}
式(\ref{eq:2.17})をよく見ればニュートンの第二法則と同じになっていることに気付く.というのも$x\rightarrow t$,場$\phi$を質点(analogue particle)の位置座標と捉えればこれは単なるポテンシャルが$-U(\phi)$の運動方程式である.このことからこの運動の全エネルギーは
\begin{align}
    W \equiv \frac{1}{2}\left(\frac{\mathrm{d} \phi}{\mathrm{d}x}\right)^{2}-U(\phi)\label{eq:2.18}
\end{align}
となることがわかる.ここで先に述べた境界条件を課せば$W=0$のゼロエネルギーとなる.また式(\ref{eq:2.14})で与えられる実際の系においてのエネルギーに立ち返ってみても
\begin{align}
    E=\int_{-\infty}^{\infty}\left[\frac{1}{2}\left(\frac{\mathrm{d}\phi}{\mathrm{d}x}\right)^2+U(\phi)\right]\mathrm{d}x
\end{align}
となり,これは明らかにanalogue粒子の運動の作用積分と同じ形になっていることがわかる.したがって,この定常解は粒子のゼロエネルギー($W=0$)の軌道の有限作用に対応しているといえる.

さらに式(\ref{eq:2.17})を解いてみる.両辺に$\phi'$をかけて積分すれば
\begin{align}
    \phi'\phi''=\frac{\partial U}{\partial\phi}\nonumber                                            \\
    \Rightarrow \int\phi'\phi''\mathrm{d}x=\int\frac{\partial U}{\partial \phi}\mathrm{d}x\nonumber \\
    \Rightarrow \frac{1}{2}\left(\phi'\right)^2=U(\phi)\label{eq:2.20}
\end{align}
となる.(なおここで境界条件である$x\rightarrow \pm \infty$で$U(\phi)\rightarrow 0, (\partial \phi/\partial x)\rightarrow 0$を課すことで積分定数はゼロと定まっている.)式(\ref{eq:2.20})はまさにanalogue粒子におけるビリアル定理に対応していることがみてとれる.

\subsection{Some potential}
\subsubsection{Case of potential which have unique minimum}
先の結果を踏まえていくつかのポテンシャル上でanalogue粒子がどのような振る舞いをするかをみてみる.まず図\ref{potential1}のような$\phi=\phi_1$で最大値$U(\phi_1)=0$を取るようなポテンシャルを考える.このときanalogue粒子のゼロエネルギーの軌道はポテンシャル上で$\phi=\phi_1$で最大値0をとりその他は常に負の値をとる.ここでSolitary waveの境界条件に立ち返ってみると,十分遠方($x\rightarrow \pm \infty$)で$\phi=\phi_1$となる必要があるが,図を見てもわかるように例えば$x\rightarrow -\infty$で$\phi=\phi_1$だった場合に$x\rightarrow \infty$で$\phi=\phi_1$にもう一度なることは決してできない.というのも式(\ref{eq:2.18})において$W=0$であったから,仮に$\phi=\phi_1$から右か左のどちらかに転がっていってしまった場合常に$[-U()\phi)]$は負の値で小さくなっていくから$W=0$に保つために運動エネルギー項($1/2(\mathrm{d}\phi/\mathrm{dx})^2$)は常に正で値が大きくなる.つまり一度$\phi=\phi_1$から進んだら粒子は止まることができなくなる.つまりこのポテンシャルでは解かずしてSolitary waveにはなり得ないということがわかる.

\begin{figure}[H]
    \centering
    \includegraphics[width=5cm]{figure/potential1.png}
    \caption{(a)}
    \label{potential1}
\end{figure}

\subsubsection{Case of potential which have degenerate minima}
次に2個以上最小値が存在する例として図\ref{potential2}のようなポテンシャルを考える.今回の場合$\phi=\phi_1,\phi_2,\phi_3$の3つの場所で最小値をとる.

\begin{figure}[H]
    \centering
    \includegraphics[width=5cm]{figure/potential2.png}
    \caption{(b)}
    \label{potential2}
\end{figure}

\end{document}
